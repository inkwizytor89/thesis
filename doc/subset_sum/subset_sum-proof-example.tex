\begin{figure}[tbh]
\centering

\begin{tikzpicture} [node distance=1.5cm]

\tikzstyle{unmark_vertex}=[circle,thick,draw=blue!75,fill=blue!20,minimum size=5mm]
\tikzstyle{mark_vertex}=[rectangle,thick,draw=red!75,fill=red!20,minimum size=4mm]

  \begin{scope}[xshift=-1.5cm, yshift=4cm]
  
  	\node [mark_vertex] (v1) {$v_{1}$};
  	\node [unmark_vertex] (v2) [above of=v1] {$v_{2}$}
  		edge node[left] {$e_{1}$} (v1);
  	\node [unmark_vertex] (v3) [right of=v1] {$v_{3}$}
  		edge node[above] {$e_{2}$} (v1);
  	\node [mark_vertex] (v4) [right of=v2] {$v_{4}$}
		edge node[above] {$e_{3}$} (v2)  		
  		edge node[left] {$e_{4}$} (v3);
  	\node [unmark_vertex] (v5) [right of=v3] {$v_{5}$}
  		edge node[right] {$e_{5}$} (v4);

	\node at (1.4cm, -1.0cm) {\text{(1)}};
  \end{scope}
  
  \begin{scope}[yshift=-0.5cm]
  
    \tikzset{BarreStyle/.style =   {opacity=.4,line width=6 mm,line cap=round,color=#1}}
  	\tikzset{node style ge/.style={circle}}

    \matrix (A) [matrix of math nodes, nodes = {node style ge},,column sep=0 mm] 
	{ 
  		      & e_{5} & e_{4} & e_{3} & e_{2} & e_{1} \\
  		v_{1} & 0     & 0     & 0     & 1     & 1     \\
  		v_{2} & 0     & 0     & 1     & 0     & 1     \\
  		v_{3} & 0     & 1     & 0     & 1     & 0     \\
  		v_{4} & 1     & 1     & 1     & 0     & 0     \\
  		v_{5} & 1     & 0     & 0     & 0     & 0     \\
	};

	\draw [BarreStyle=red] (A-2-1.west) to (A-2-6.east);
	\draw [BarreStyle=red] (A-5-1.west) to (A-5-6.east);

	\node at (0cm, -2.7cm) {\text{(2)}};
  \end{scope}
  
  \begin{scope}[xshift=8cm]
  	\tikzset{BarreStyle/.style =   {opacity=.4,line width=6 mm,line cap=round,color=#1}}
  	\tikzset{LineStyle/.style =   {opacity=.7,line width=0.5 mm,line cap=round,color=#1}}
  	\tikzset{node style ge/.style={circle}}

    \matrix (A) [matrix of math nodes, nodes = {node style ge},,column sep=0 mm] 
	{ 
  		x_{1}  & =  & 1  & 0  & 0  & 0  & 1  & 1  & =  & 1029 \\
  		x_{2}  & =  & 1  & 0  & 0  & 1  & 0  & 1  & =  & 1041 \\
  		x_{3}  & =  & 1  & 0  & 1  & 0  & 1  & 0  & =  & 1092 \\
  		x_{4}  & =  & 1  & 1  & 1  & 1  & 0  & 0  & =  & 1360 \\
  		x_{5}  & =  & 1  & 1  & 0  & 0  & 0  & 0  & =  & 1280 \\
  		y_{1}  & =  & 0  & 0  & 0  & 0  & 0  & 1  & =  & 1    \\
  		y_{2}  & =  & 0  & 0  & 0  & 0  & 1  & 0  & =  & 4    \\
  		y_{3}  & =  & 0  & 0  & 0  & 1  & 0  & 0  & =  & 16   \\
  		y_{4}  & =  & 0  & 0  & 1  & 0  & 0  & 0  & =  & 64   \\
  		y_{5}  & =  & 0  & 1  & 0  & 0  & 0  & 0  & =  & 256  \\
  		   -   & -  &  - & -  & -  & -  & -  & -  & -  &  -   \\
  		t  & =  & 2  & 2  & 2  & 2  & 2  & 2  & =  & 2730     \\
	};
	
	\draw [BarreStyle=red] (A-1-1.west) to (A-1-10.east);
	\draw [BarreStyle=red] (A-4-1.west) to (A-4-10.east);
	\draw [BarreStyle=red] (A-6-1.west) to (A-6-10.east);
	\draw [BarreStyle=red] (A-7-1.west) to (A-7-10.east);
	\draw [BarreStyle=red] (A-8-1.west) to (A-8-10.east);
	\draw [BarreStyle=red] (A-9-1.west) to (A-9-10.east);
	\draw [BarreStyle=red] (A-10-1.west) to (A-10-10.east);
	\draw [LineStyle=black] (A-11-1.west) to (A-11-10.east);
	
	\node at (0cm, -7cm) {\text{(3)}};
	
  \end{scope}

\end{tikzpicture}
\caption{Redukcja problemu VERTEX-COVER do problemu SUBSET-SUM.
	\newline (1) Przykład grafu nieskierowanego. Wierzchołki minimalnego pokrycia wierzchołkowego prezentowane są na czerwono
	\newline (2) Macierz incydencji dla przykładowego grafu. Kolorem czerwonym zostały oznaczone wierzchołki z minimalnego pokrycia wierzchołkowego.
	\newline (3) Rozwiązanie problemu SUBSET-SUM odpowiadające przykładowemu grafu dla problemu VERTEX-COVER. Elementy typu $x$ odpowiadają wierzchołką, a kolorem oznaczone są te których wierzchołki wchodzą do pokrycia wierzchołkowego. Elementy typu $y$ którym odpowiadają krawędzie, kolorem czerwonym oznaczone te które mają koniec w dokładnie jednym wierzchołku pokrycia wierzchołkowego. Najbardziej znaczący bit tworzący sumę $t$ jest równy rozmiarowi pokrycia wierzchołkowego, zaś reszta bitów ma wartość 2. Wszystkie elementy tworzą zbiór $S$, natomiaste te oznaczone kolorem czerwoną tworzą szukaną sumę.
	 Elementy oznaczone
	}
\label{subset_sum-proof-example}
\end{figure}
