\chapter{Problem sumy podzbioru}

\section{Wstęp}

Ostatnim problem który zostanie poruszony w tej pracy jest problem sumy podzbioru. Problem ten jest natury numerycznej. W tym problemie należy skonstruować szukaną sumę w oparciu o liczby z dostępnego zbioru. Jest to kolejny problem który można łatwo określić, a przez to podejrzewać że istnieje prosty algorytm rozwiązujący ten problem. Jednak zostanie przedstawione jest to problem NP-zupełny.

\section{Przedstawienie problemu}

Problem sumy podzbioru(SUBSET-SUM) polega na wyznaczeniu podzbioru $S' \subseteq S$ dla którego suma elementów jest maksymalna, ale nie większa od zadanej wartości $t$.


\begin{twr}
SUBSET-SUM $\in$ NP-zupełnych.
\end{twr}

\section{Dowód przez redukcje}
\begin{proof}
Problem SUBSET-SUM $\in$ NP, ponieważ dla zadanego zbioru liczb $S' \subseteq S$ możemy w czasie liniowym zsumować je a otrzymaną sumę porównać z wartoścą $t$.

Następnie zostanie pokazana redukcja z problemu VERTEX-COVER do problemu SUBSET-SUM, czyli VERTEX-COVER < = p SUBSET-SUM. Co udowodni że SUBSET-SUM $\in$ NPC. 

Dla dowolnego egzemplarza <G,k> problemu pokrycia wierchołkowego zostanie przedstawiona konstrukcja odpowiadającego mu egzemplarza <S,t> problemu sumy podzbioru. 

Dla dowolnego grafu G(V,E) skonstujmy macierz T wymiaru |E+1| * |V + E| której wiersze będą odzwierciedlały kolejne elementy zbioru S zakodowane w układzie czwórkowym. Macierz T składa się z dwóch grup $x_i$ oraz $y_i$.

Grupę $x_i$ determinują wierzchołki grafu i jest ona równoliczna ze zbiorem wierzchołków |V|. Grupę tę wypełnia macierz incydencji M grafu G od strony najmniej znaczącego bitu, natomiast kolumna z najbardziej znaczącym jest wypełniona przez wartość 1.
[wzór na element]

Grupa $y_i$ to macież jednostkowa $I_{|E|}$ począwszy od najmniej znaczącego bitu, a wypełniona wartością 0 na pozycji najbardziej zaczącego bitu.
[wzór na element]

Ostatnim elementem konstrukcji jest wyznaczenie wartości $t$:

 $ t = k4^{|E|} + sum (od j=0, do |E|-1) 2 * 4^j$. 

Elementy składające się na sumę $t$ z grupy $x_i$ to te $x_i$ którym odpowiadają wierzchołki wchodzące do pokrycia wierzchołkowego. Natomiast z grupy $y_i$ są to te $y_i$ odpowiadające krawędzią dla których dokładnie jeden z końców znajduje się w pokryciu wierzchołkowym.

\begin{figure}[tbh]
\centering

\begin{tikzpicture} [node distance=1.5cm]

\tikzstyle{unmark_vertex}=[circle,thick,draw=blue!75,fill=blue!20,minimum size=5mm]
\tikzstyle{mark_vertex}=[rectangle,thick,draw=red!75,fill=red!20,minimum size=4mm]

  \begin{scope}[xshift=-1.5cm, yshift=4cm]
  
  	\node [mark_vertex] (v1) {$v_{1}$};
  	\node [unmark_vertex] (v2) [above of=v1] {$v_{2}$}
  		edge node[left] {$e_{1}$} (v1);
  	\node [unmark_vertex] (v3) [right of=v1] {$v_{3}$}
  		edge node[above] {$e_{2}$} (v1);
  	\node [mark_vertex] (v4) [right of=v2] {$v_{4}$}
		edge node[above] {$e_{3}$} (v2)  		
  		edge node[left] {$e_{4}$} (v3);
  	\node [unmark_vertex] (v5) [right of=v3] {$v_{5}$}
  		edge node[right] {$e_{5}$} (v4);

	\node at (1.4cm, -1.0cm) {\text{(1)}};
  \end{scope}
  
  \begin{scope}[yshift=-0.5cm]
  
    \tikzset{BarreStyle/.style =   {opacity=.4,line width=6 mm,line cap=round,color=#1}}
  	\tikzset{node style ge/.style={circle}}

    \matrix (A) [matrix of math nodes, nodes = {node style ge},,column sep=0 mm] 
	{ 
  		      & e_{5} & e_{4} & e_{3} & e_{2} & e_{1} \\
  		v_{1} & 0     & 0     & 0     & 1     & 1     \\
  		v_{2} & 0     & 0     & 1     & 0     & 1     \\
  		v_{3} & 0     & 1     & 0     & 1     & 0     \\
  		v_{4} & 1     & 1     & 1     & 0     & 0     \\
  		v_{5} & 1     & 0     & 0     & 0     & 0     \\
	};

	\draw [BarreStyle=red] (A-2-1.west) to (A-2-6.east);
	\draw [BarreStyle=red] (A-5-1.west) to (A-5-6.east);

	\node at (0cm, -2.7cm) {\text{(2)}};
  \end{scope}
  
  \begin{scope}[xshift=8cm]
  	\tikzset{BarreStyle/.style =   {opacity=.4,line width=6 mm,line cap=round,color=#1}}
  	\tikzset{LineStyle/.style =   {opacity=.7,line width=0.5 mm,line cap=round,color=#1}}
  	\tikzset{node style ge/.style={circle}}

    \matrix (A) [matrix of math nodes, nodes = {node style ge},,column sep=0 mm] 
	{ 
  		x_{1}  & =  & 1  & 0  & 0  & 0  & 1  & 1  & =  & 1029 \\
  		x_{2}  & =  & 1  & 0  & 0  & 1  & 0  & 1  & =  & 1041 \\
  		x_{3}  & =  & 1  & 0  & 1  & 0  & 1  & 0  & =  & 1092 \\
  		x_{4}  & =  & 1  & 1  & 1  & 1  & 0  & 0  & =  & 1360 \\
  		x_{5}  & =  & 1  & 1  & 0  & 0  & 0  & 0  & =  & 1280 \\
  		y_{1}  & =  & 0  & 0  & 0  & 0  & 0  & 1  & =  & 1    \\
  		y_{2}  & =  & 0  & 0  & 0  & 0  & 1  & 0  & =  & 4    \\
  		y_{3}  & =  & 0  & 0  & 0  & 1  & 0  & 0  & =  & 16   \\
  		y_{4}  & =  & 0  & 0  & 1  & 0  & 0  & 0  & =  & 64   \\
  		y_{5}  & =  & 0  & 1  & 0  & 0  & 0  & 0  & =  & 256  \\
  		   -   & -  &  - & -  & -  & -  & -  & -  & -  &  -   \\
  		t  & =  & 2  & 2  & 2  & 2  & 2  & 2  & =  & 2730     \\
	};
	
	\draw [BarreStyle=red] (A-1-1.west) to (A-1-10.east);
	\draw [BarreStyle=red] (A-4-1.west) to (A-4-10.east);
	\draw [BarreStyle=red] (A-6-1.west) to (A-6-10.east);
	\draw [BarreStyle=red] (A-7-1.west) to (A-7-10.east);
	\draw [BarreStyle=red] (A-8-1.west) to (A-8-10.east);
	\draw [BarreStyle=red] (A-9-1.west) to (A-9-10.east);
	\draw [BarreStyle=red] (A-10-1.west) to (A-10-10.east);
	\draw [LineStyle=black] (A-11-1.west) to (A-11-10.east);
	
	\node at (0cm, -7cm) {\text{(3)}};
	
  \end{scope}

\end{tikzpicture}
\caption{Redukcja problemu VERTEX-COVER do problemu SUBSET-SUM.
	\newline (1) Przykład grafu nieskierowanego. Wierzchołki minimalnego pokrycia wierzchołkowego prezentowane są na czerwono
	\newline (2) Macierz incydencji dla przykładowego grafu. Kolorem czerwonym zostały oznaczone wierzchołki z minimalnego pokrycia wierzchołkowego.
	\newline (3) Rozwiązanie problemu SUBSET-SUM odpowiadające przykładowemu grafu dla problemu VERTEX-COVER. Elementy typu $x$ odpowiadają wierzchołką, a kolorem oznaczone są te których wierzchołki wchodzą do pokrycia wierzchołkowego. Elementy typu $y$ którym odpowiadają krawędzie, kolorem czerwonym oznaczone te które mają koniec w dokładnie jednym wierzchołku pokrycia wierzchołkowego. Najbardziej znaczący bit tworzący sumę $t$ jest równy rozmiarowi pokrycia wierzchołkowego, zaś reszta bitów ma wartość 2. Wszystkie elementy tworzą zbiór $S$, natomiaste te oznaczone kolorem czerwoną tworzą szukaną sumę.
	 Elementy oznaczone
	}
\label{subset_sum-proof-example}
\end{figure}


Przy tak określonej redukcji zostanie pokazane, że graf G będzie miał pokrycie wierzchołkowe k wtedy i tylko wtedy, gdy zbiór S będzie zawierał część elementów sumujących się do t.

W pierwszej kolejności załóżmy, że graf G ma pokrycie wierzchołkowe $ { v_1,v_2, v_3 ... wiekszy odstęp  v_k} \subset $ V. Na podstawie grafu G otrzymujemy macierz w oparciu o przedstawioną wyżej konstrukcje. Z grupy X wybrane są elementy $x_i$ odpowiadającę pokryciu wierzchołkowemu, natomiast z grupy Y te $y_i$ których odpowiadające krawędzie mają koniec w dokłdnie jednym wierchołku pokrycia wierzchołkowemu. Udowodnijmy dalej że po wybraniu dokładnie tych elementów suma ich będzię równa zdefiniowanej przez nas wartośći $t$. W grupie X wartość najbardziej znaczącego bitu wynosi dokładnie k elementów, ponieważ one zostały wybrane z pokrycia wierzchołkowego. Natomiast w grupie Y na najbardziej znaczącym bicie występują zawsze wartości 0. Pozostałe bity zdeterminowane są przez krawędzie w grafie, a wartość $t$ w każdej kolumnie wynosi $2$. Każda kolumna w macierzy odpowiada krawędzi grafu, zatem w każdej kolumnie w grupie X będą dokładnie dwie wartości 1. Jednak uwzględniajac tylko wierzchołki z pokrycia, to w każdej kolumnie może występować conajmniej raz warość 1. Spoglądając szerzej jeśli krawędz jest pokryta wyłącznie przez jeden wierzchołek to w kolumnie znajduje się wyłącznie jedna wartość 1. W takim przypadku do zbioru $S'$ dodajemy jednak element z grupy Y, który w tej pozycji ma wartość 1. W innym przypadku jak krawędź jest pokryta przez dwa wierzchołki to w grupie X w odpowiedniej kolumnie występują dokładnie dwie wartości 1. Dla takiego przypadku nie zostaje dodany odpowiadający mu element z grupy Y. Podsumowując w kolumnie najbardziej suma bitów zawsze wynosi k, a w pozostałych kolumnach suma zawsze wynosi 2. Oznacza to że elementy ze zbioru $S'$ zawsze będą się w ten sposób sumowały do warości $t$.

Teraz udowodnijmy przeciwne wnioskowanie i załóży że istnieje taki podzbiór $S' zawierasie S$, którego elementy sumują się do wartosci $t$.  Wartość $t$ zdefiniowana jest w ten sposób że składa się z wartości 2 na każdym bicie w kodzie czwórkowym poza najbardziej znaczącym bitem. Najbardziej znaczący bit oznaczmy jako $m$. Dzięki temu można odczytać ile krawędzi znajduje się w grafie G. Dalej podzielmy podzbiór $S'$ na dwie grupy. Grupa Y zawiera podzbiory $ S' $, których suma wynosi wartość potęgi liczby 4. Pozostałe elementy stanowią podzbiory których suma należy do grupy X, podzbiory charakteryzuje to że ich najbardziej znaczący bit jest związany z taką potęgą liczby 4 co wartość $m$. Dzięki temu możemy podzielić zbiór $ S' $ w taki sposób $ { x_i1, x_i2, ... x_im } suma { y_j1, y_j2, ... y_jp } $ . Dodatkowo zbiór $ S $ zawiera dodatkowe elementy które wypełną wiersze macierzy z konstrukcji. W rezultacie w każdej kolumnie poza najbardziej znaczącym bitem muszą znajdować się trzy wartości 1, dwie w grupie X i jedna w grupie Y, tak by krawędź w grafie miała dokładnie 2 końce. Dzięki temu że elementy z $S'$ sumują się do $t$ oznacza to że każdemu elementow z $ {x_i1, x_i2, ... x_im}$ odpowiada wierchołkowi z pokrycie wierzchołkoweg ponieważ w kolumnie mniejznaczących bitów każda wartość wynosi 2. W rezultacie $m = k$ i kończy to dowód. 


Pokażemy że $m = k$ oraz że elementą $x_i$ będą odpowiadały wierzchołki $v_i$ tworzące pokrycie wierzchołkowe w grafie G.


W takim przypadku aby można było otrzymać poprawny graf w grupie X musi znajdować się conajmniej $m$ elementów, tak by suma najbardziej znaczących bitów odpowiadała wartości $t$. W przypadku mniej znaczących bitów w każdej kolumnie muszą znajdować się trzy warości 1. 


W tak zdefiniowanym zbiorze odpowiadającym zasadą konstrukcji wszystkie elementy możemy przedstawić w systemie czwórkowym


Ciąg dalszy nastąpi... że redukcja jest wielomianowa.


Zbiór $S'$ tworzą liczby z pokrycia wierzchołkowego, zatem każda krawędz jest pokryta przez 2 lub 1 wierzchołek. 

Wybrane wiersze sumowały się do wartości $2$ w kolumnie 



 Liczbę $t$ tworzą dwie grupy X i Y.
\end{proof}

\section{Algorytm aproksymacyjny}