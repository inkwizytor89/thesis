%\usepackage[OT4]{polski}	% tryb pelnej polonizacji

\usepackage[utf8]{inputenc}	% kodowanie
\usepackage[T1]{fontenc}
\usepackage[MeX]{polski}
%\usepackage{amsrefs}					% bibliografia
\usepackage{makeidx}					% indeks
\usepackage[pdftex]{graphicx}	% zalaczanie grafiki

\usepackage{tikz}							% grafika wektorowa
\usetikzlibrary{arrows,shapes,snakes,automata,backgrounds,petri}

\usepackage{setspace}					% interlinia

\usepackage{hyperref}				% wewnetrzne odnosniki w dokumencie

%		te rozwazam:
\usepackage{listings}					% kody zrodlowe
\usepackage{fancyhdr}					%	zywe paginy smierci

\usepackage{tocloft}					% format spisu tresci
%\usepackage{array}						% ladniejsze tabelki
\usepackage{multirow}					% laczenie wierszy w tabelach
\usepackage[tableposition=top,format=hang,labelsep=period,labelfont={bf,small},textfont=small]{caption}	
															% formatuje podpisy pod rysunkami i tabelami, format=hang powoduje,
															% ze kolejne linie podpisu beda wciete az do odleglosci nazwy podpisu np. "Rysunek 1."
\usepackage{floatflt}					% ladne oplywanie obrazkow tekstem
\usepackage{url}						% url w bibliografii
\usepackage{amsmath} 

%pseudokod
\usepackage{amsmath}
\usepackage{algorithm}
\usepackage[noend]{algpseudocode}

%domyslnie rozdzial nie zaczynaja sie od akapitow, ale w Polsce taki zwyczaj jest
\usepackage{indentfirst}

\usepackage{multicol} %pozwala na pisanie tekstu w kilku kolumnach

\usepackage{amssymb} %pozwala na fajne zapisanie np. zbioru liczb rzeczywistych ($\mathbb{R}^{3}$)
\usepackage{amsthm}