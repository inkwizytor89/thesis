\chapter*{Wstęp}
\markboth{Wstęp}{Wstęp}
\addcontentsline{toc}{chapter}{Wstęp}

Zasadniczym celem pracy dyplomowej przygotowywanej przez studenta jest:

\begin{itemize}
\item wykazanie sie umiejętnością formułowania i rozwiązywania problemów wiążących się z programem odbytych studiów,
\item wykazanie sie znajomością metod i sposobów prowadzenia analizy oraz redakcyjnego przygotowania pracy w oparciu o umiejętności nabyte w czasie studiów.
\end {itemize}

Praca dyplomowa musi być samodzielnym opracowaniem autorstwa studenta, przygotowanym przy pomocy promotora. Student, jako autor ponosi pełna odpowiedzialność z tytułu oryginalności i rzetelności zaprezentowanego materiału i powinien uwzględniać wszelkie prawa i dobre obyczaje w tym zakresie.

Tematyka pracy powinna znacząco wykraczać poza materiał omówiony w trakcie studiów; zakłada samodzielny wkład autora w postaci np.: implementacji, opracowania algorytmu, samodzielnego porównania, oceny i analizy istniejących rozwiązań.



