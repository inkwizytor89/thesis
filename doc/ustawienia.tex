% ------------------------------------------------------------------------
%   Kropki po numerach sekcji, podsekcji, itd.
%   Np. 1.2. Tytul podrozdzialu
% ------------------------------------------------------------------------
\makeatletter
    \def\numberline#1{\hb@xt@\@tempdima{#1.\hfil}}                      %kropki w spisie tresci
    \renewcommand*\@seccntformat[1]{\csname the#1\endcsname.\enspace}   %kropki w tresci dokumentu
\makeatother

%numerowanie tabel:
%\renewcommand{\thetable}{\thechapter.\arabic{figure}}


%
%twierdzenia, definicje i lematy
%
\newtheorem{defin}{Definicja}
\newtheorem{twr}{Twierdzenie}
\newtheorem{lem}[twr]{Lemat}
\renewenvironment{proof}{Dowód:}
% ------------------------------------------------------------------------
% Inne
% ------------------------------------------------------------------------
\frenchspacing
\setlength{\parskip}{3pt}           	% odstep pom. akapitamia
\linespread{1.49}                    	% odstep pomiedzy liniami (interlinia)
%\onehalfspacing

\setcounter{tocdepth}{2}            	% stopien zaglebienia w spisie tresci
\setcounter{secnumdepth}{2}         	% do jakiego stopnia zaglebienia numeracja

% polskie podpisy
\renewcommand{\figurename}{Rys.}
\renewcommand{\tablename}{Tab.}

%paginy zywe smierci
\pagestyle{fancy}
% zmiana liter w~zywej paginie na małe
\renewcommand{\chaptermark}[1]{\markboth{#1}{}}
\renewcommand{\sectionmark}[1]{\markright{\thesection\ #1}}
\fancyhf{} % usun biezace ustawienia pagin
\fancyhead[LE,RO]{\small\bfseries\thepage}
\fancyhead[LO]{\small\bfseries\rightmark}
\fancyhead[RE]{\small\bfseries\leftmark}
\renewcommand{\headrulewidth}{0.5pt}
\renewcommand{\footrulewidth}{0pt}
\addtolength{\headheight}{0.5pt} % pionowy odstep na kreske
\fancypagestyle{plain}{%
\fancyhead{} % usun p. górne na stronach pozbawionych
% numeracji (plain)
\renewcommand{\headrulewidth}{0pt} % pozioma kreska
} 